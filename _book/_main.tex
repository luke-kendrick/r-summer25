% Options for packages loaded elsewhere
\PassOptionsToPackage{unicode}{hyperref}
\PassOptionsToPackage{hyphens}{url}
%
\documentclass[
]{book}
\usepackage{amsmath,amssymb}
\usepackage{iftex}
\ifPDFTeX
  \usepackage[T1]{fontenc}
  \usepackage[utf8]{inputenc}
  \usepackage{textcomp} % provide euro and other symbols
\else % if luatex or xetex
  \usepackage{unicode-math} % this also loads fontspec
  \defaultfontfeatures{Scale=MatchLowercase}
  \defaultfontfeatures[\rmfamily]{Ligatures=TeX,Scale=1}
\fi
\usepackage{lmodern}
\ifPDFTeX\else
  % xetex/luatex font selection
\fi
% Use upquote if available, for straight quotes in verbatim environments
\IfFileExists{upquote.sty}{\usepackage{upquote}}{}
\IfFileExists{microtype.sty}{% use microtype if available
  \usepackage[]{microtype}
  \UseMicrotypeSet[protrusion]{basicmath} % disable protrusion for tt fonts
}{}
\makeatletter
\@ifundefined{KOMAClassName}{% if non-KOMA class
  \IfFileExists{parskip.sty}{%
    \usepackage{parskip}
  }{% else
    \setlength{\parindent}{0pt}
    \setlength{\parskip}{6pt plus 2pt minus 1pt}}
}{% if KOMA class
  \KOMAoptions{parskip=half}}
\makeatother
\usepackage{xcolor}
\usepackage{color}
\usepackage{fancyvrb}
\newcommand{\VerbBar}{|}
\newcommand{\VERB}{\Verb[commandchars=\\\{\}]}
\DefineVerbatimEnvironment{Highlighting}{Verbatim}{commandchars=\\\{\}}
% Add ',fontsize=\small' for more characters per line
\usepackage{framed}
\definecolor{shadecolor}{RGB}{248,248,248}
\newenvironment{Shaded}{\begin{snugshade}}{\end{snugshade}}
\newcommand{\AlertTok}[1]{\textcolor[rgb]{0.94,0.16,0.16}{#1}}
\newcommand{\AnnotationTok}[1]{\textcolor[rgb]{0.56,0.35,0.01}{\textbf{\textit{#1}}}}
\newcommand{\AttributeTok}[1]{\textcolor[rgb]{0.13,0.29,0.53}{#1}}
\newcommand{\BaseNTok}[1]{\textcolor[rgb]{0.00,0.00,0.81}{#1}}
\newcommand{\BuiltInTok}[1]{#1}
\newcommand{\CharTok}[1]{\textcolor[rgb]{0.31,0.60,0.02}{#1}}
\newcommand{\CommentTok}[1]{\textcolor[rgb]{0.56,0.35,0.01}{\textit{#1}}}
\newcommand{\CommentVarTok}[1]{\textcolor[rgb]{0.56,0.35,0.01}{\textbf{\textit{#1}}}}
\newcommand{\ConstantTok}[1]{\textcolor[rgb]{0.56,0.35,0.01}{#1}}
\newcommand{\ControlFlowTok}[1]{\textcolor[rgb]{0.13,0.29,0.53}{\textbf{#1}}}
\newcommand{\DataTypeTok}[1]{\textcolor[rgb]{0.13,0.29,0.53}{#1}}
\newcommand{\DecValTok}[1]{\textcolor[rgb]{0.00,0.00,0.81}{#1}}
\newcommand{\DocumentationTok}[1]{\textcolor[rgb]{0.56,0.35,0.01}{\textbf{\textit{#1}}}}
\newcommand{\ErrorTok}[1]{\textcolor[rgb]{0.64,0.00,0.00}{\textbf{#1}}}
\newcommand{\ExtensionTok}[1]{#1}
\newcommand{\FloatTok}[1]{\textcolor[rgb]{0.00,0.00,0.81}{#1}}
\newcommand{\FunctionTok}[1]{\textcolor[rgb]{0.13,0.29,0.53}{\textbf{#1}}}
\newcommand{\ImportTok}[1]{#1}
\newcommand{\InformationTok}[1]{\textcolor[rgb]{0.56,0.35,0.01}{\textbf{\textit{#1}}}}
\newcommand{\KeywordTok}[1]{\textcolor[rgb]{0.13,0.29,0.53}{\textbf{#1}}}
\newcommand{\NormalTok}[1]{#1}
\newcommand{\OperatorTok}[1]{\textcolor[rgb]{0.81,0.36,0.00}{\textbf{#1}}}
\newcommand{\OtherTok}[1]{\textcolor[rgb]{0.56,0.35,0.01}{#1}}
\newcommand{\PreprocessorTok}[1]{\textcolor[rgb]{0.56,0.35,0.01}{\textit{#1}}}
\newcommand{\RegionMarkerTok}[1]{#1}
\newcommand{\SpecialCharTok}[1]{\textcolor[rgb]{0.81,0.36,0.00}{\textbf{#1}}}
\newcommand{\SpecialStringTok}[1]{\textcolor[rgb]{0.31,0.60,0.02}{#1}}
\newcommand{\StringTok}[1]{\textcolor[rgb]{0.31,0.60,0.02}{#1}}
\newcommand{\VariableTok}[1]{\textcolor[rgb]{0.00,0.00,0.00}{#1}}
\newcommand{\VerbatimStringTok}[1]{\textcolor[rgb]{0.31,0.60,0.02}{#1}}
\newcommand{\WarningTok}[1]{\textcolor[rgb]{0.56,0.35,0.01}{\textbf{\textit{#1}}}}
\usepackage{longtable,booktabs,array}
\usepackage{calc} % for calculating minipage widths
% Correct order of tables after \paragraph or \subparagraph
\usepackage{etoolbox}
\makeatletter
\patchcmd\longtable{\par}{\if@noskipsec\mbox{}\fi\par}{}{}
\makeatother
% Allow footnotes in longtable head/foot
\IfFileExists{footnotehyper.sty}{\usepackage{footnotehyper}}{\usepackage{footnote}}
\makesavenoteenv{longtable}
\usepackage{graphicx}
\makeatletter
\newsavebox\pandoc@box
\newcommand*\pandocbounded[1]{% scales image to fit in text height/width
  \sbox\pandoc@box{#1}%
  \Gscale@div\@tempa{\textheight}{\dimexpr\ht\pandoc@box+\dp\pandoc@box\relax}%
  \Gscale@div\@tempb{\linewidth}{\wd\pandoc@box}%
  \ifdim\@tempb\p@<\@tempa\p@\let\@tempa\@tempb\fi% select the smaller of both
  \ifdim\@tempa\p@<\p@\scalebox{\@tempa}{\usebox\pandoc@box}%
  \else\usebox{\pandoc@box}%
  \fi%
}
% Set default figure placement to htbp
\def\fps@figure{htbp}
\makeatother
\setlength{\emergencystretch}{3em} % prevent overfull lines
\providecommand{\tightlist}{%
  \setlength{\itemsep}{0pt}\setlength{\parskip}{0pt}}
\setcounter{secnumdepth}{5}
\usepackage{booktabs}
\usepackage[]{natbib}
\bibliographystyle{plainnat}
\usepackage{bookmark}
\IfFileExists{xurl.sty}{\usepackage{xurl}}{} % add URL line breaks if available
\urlstyle{same}
\hypersetup{
  pdftitle={R Training Summer 2025},
  pdfauthor={Luke Kendrick / Victoria Bourne},
  hidelinks,
  pdfcreator={LaTeX via pandoc}}

\title{R Training Summer 2025}
\author{Luke Kendrick / Victoria Bourne}
\date{2025-07-14}

\begin{document}
\maketitle

{
\setcounter{tocdepth}{1}
\tableofcontents
}
\chapter*{Psychology @ Royal Holloway University of London R Training}\label{psychology-royal-holloway-university-of-london-r-training}
\addcontentsline{toc}{chapter}{Psychology @ Royal Holloway University of London R Training}

Summer/Autumn Sessions 2025

\chapter{Session 1: Starting with R}\label{session-1-starting-with-r}

\section{Download Slides and Data Files}\label{download-slides-and-data-files}

Slides: \href{downloads/startr.pdf}{here}.

Data file: height.csv \href{downloads/height.csv}{here}.

Data file sleep.csv \href{downloads/sleep.csv}{here}.

Data file: intervention.csv \href{downloads/intervention.csv}{here}.

\section{Code Used Below:}\label{code-used-below}

This is the code used from slide 42 onwards:

\subsection{Install and Load Package:}\label{install-and-load-package}

\begin{Shaded}
\begin{Highlighting}[]
\FunctionTok{install.packages}\NormalTok{(}\StringTok{"tidyverse"}\NormalTok{) }\CommentTok{\# install only if needed}
\FunctionTok{library}\NormalTok{(tidyverse) }\CommentTok{\# always load the package before starting}
\end{Highlighting}
\end{Shaded}

Code for the first data file:

\begin{Shaded}
\begin{Highlighting}[]
\NormalTok{data }\OtherTok{\textless{}{-}} \FunctionTok{read\_csv}\NormalTok{(}\StringTok{"height.csv"}\NormalTok{)}

\FunctionTok{print}\NormalTok{(data)}
\FunctionTok{view}\NormalTok{(data)}
\end{Highlighting}
\end{Shaded}

\subsection{Calculating Descriptive Statistics}\label{calculating-descriptive-statistics}

\begin{Shaded}
\begin{Highlighting}[]
\FunctionTok{rm}\NormalTok{(data) }\CommentTok{\# will remove the object called “data”.}

\NormalTok{data }\OtherTok{\textless{}{-}} \FunctionTok{read\_csv}\NormalTok{(“sleep.csv”) }\CommentTok{\# new data set}
\end{Highlighting}
\end{Shaded}

Explore the new data set:

\begin{Shaded}
\begin{Highlighting}[]
\FunctionTok{head}\NormalTok{(data) }\CommentTok{\#view the first few rows}
\FunctionTok{summary}\NormalTok{(data) }\CommentTok{\#quick summary of the data set}
\FunctionTok{names}\NormalTok{(data) }\CommentTok{\#check variable names}
\end{Highlighting}
\end{Shaded}

Count and pipe \texttt{\%\textgreater{}\%}:

\begin{Shaded}
\begin{Highlighting}[]
\NormalTok{data }\SpecialCharTok{\%\textgreater{}\%}
    \FunctionTok{count}\NormalTok{(condition) }
\end{Highlighting}
\end{Shaded}

Means and Standard Deviations:

\begin{Shaded}
\begin{Highlighting}[]
\NormalTok{descr }\OtherTok{\textless{}{-}}\NormalTok{ data }\SpecialCharTok{\%\textgreater{}\%}  
    \FunctionTok{summarise}\NormalTok{(}\AttributeTok{mean\_age =} \FunctionTok{mean}\NormalTok{(age),}
              \AttributeTok{sd\_age =} \FunctionTok{sd}\NormalTok{(age),}
              \AttributeTok{mean\_change =} \FunctionTok{mean}\NormalTok{(change))}

\FunctionTok{view}\NormalTok{(desc) }\CommentTok{\# view the descriptives}
\end{Highlighting}
\end{Shaded}

Add standard deviation for \texttt{change}:

\begin{Shaded}
\begin{Highlighting}[]
\NormalTok{descr }\OtherTok{\textless{}{-}}\NormalTok{ data }\SpecialCharTok{\%\textgreater{}\%}  
    \FunctionTok{summarise}\NormalTok{(}\AttributeTok{mean\_age =} \FunctionTok{mean}\NormalTok{(age),}
              \AttributeTok{sd\_age =} \FunctionTok{sd}\NormalTok{(age),}
              \AttributeTok{mean\_change =} \FunctionTok{mean}\NormalTok{(change)}
              \AttributeTok{sd\_change =} \FunctionTok{sd}\NormalTok{(change))}

\FunctionTok{view}\NormalTok{(desc) }\CommentTok{\# view the updated descriptives}
\end{Highlighting}
\end{Shaded}

Using \texttt{group\_by()}:

\begin{Shaded}
\begin{Highlighting}[]
\NormalTok{descr }\OtherTok{\textless{}{-}}\NormalTok{ data }\SpecialCharTok{\%\textgreater{}\%}
  \FunctionTok{group\_by}\NormalTok{(condition) }\SpecialCharTok{\%\textgreater{}\%}
    \FunctionTok{summarise}\NormalTok{(}\AttributeTok{mean\_age =} \FunctionTok{mean}\NormalTok{(age),}
              \AttributeTok{sd\_age =} \FunctionTok{sd}\NormalTok{(age),}
              \AttributeTok{mean\_change =} \FunctionTok{mean}\NormalTok{(change)}
              \AttributeTok{sd\_change =} \FunctionTok{sd}\NormalTok{(change))}
\end{Highlighting}
\end{Shaded}

\subsection{Distributions}\label{distributions}

Histogram:

\begin{Shaded}
\begin{Highlighting}[]
\FunctionTok{ggplot}\NormalTok{(data, }\FunctionTok{aes}\NormalTok{(}\AttributeTok{x =}\NormalTok{ change, }\AttributeTok{fill =}\NormalTok{ condition)) }\SpecialCharTok{+}
      \FunctionTok{geom\_histogram}\NormalTok{(}\AttributeTok{colour =} \StringTok{"black"}\NormalTok{)}
\end{Highlighting}
\end{Shaded}

\begin{itemize}
\tightlist
\item
  \texttt{facet\_wrap()}
\end{itemize}

\begin{Shaded}
\begin{Highlighting}[]
\FunctionTok{ggplot}\NormalTok{(data, }\FunctionTok{aes}\NormalTok{(}\AttributeTok{x =}\NormalTok{ change, }\AttributeTok{fill =}\NormalTok{ condition)) }\SpecialCharTok{+}
      \FunctionTok{geom\_histogram}\NormalTok{(}\AttributeTok{colour =} \StringTok{"black"}\NormalTok{) }\SpecialCharTok{+}
      \FunctionTok{facet\_wrap}\NormalTok{(}\SpecialCharTok{\textasciitilde{}}\NormalTok{ condition)}
\end{Highlighting}
\end{Shaded}

Density Plot:

\begin{Shaded}
\begin{Highlighting}[]
\FunctionTok{ggplot}\NormalTok{(data, }\FunctionTok{aes}\NormalTok{(}\AttributeTok{x =}\NormalTok{ change, }\AttributeTok{fill =}\NormalTok{ condition)) }\SpecialCharTok{+}
      \FunctionTok{geom\_density}\NormalTok{(}\AttributeTok{alpha =}\NormalTok{ .}\DecValTok{5}\NormalTok{)}
\end{Highlighting}
\end{Shaded}

\begin{itemize}
\tightlist
\item
  \texttt{facet\_wrap()}
\end{itemize}

\begin{Shaded}
\begin{Highlighting}[]
\FunctionTok{ggplot}\NormalTok{(data, }\FunctionTok{aes}\NormalTok{(}\AttributeTok{x =}\NormalTok{ change, }\AttributeTok{fill =}\NormalTok{ condition)) }\SpecialCharTok{+}
      \FunctionTok{geom\_density}\NormalTok{(}\AttributeTok{alpha =}\NormalTok{ .}\DecValTok{5}\NormalTok{)}
      \FunctionTok{facet\_wrap}\NormalTok{(}\SpecialCharTok{\textasciitilde{}}\NormalTok{ condition)}
\end{Highlighting}
\end{Shaded}

Box Plot:

\begin{Shaded}
\begin{Highlighting}[]
\FunctionTok{ggplot}\NormalTok{(data, }\FunctionTok{aes}\NormalTok{(}\AttributeTok{x =}\NormalTok{ condition, }\AttributeTok{y =}\NormalTok{ change)) }\SpecialCharTok{+}
      \FunctionTok{geom\_boxplot}\NormalTok{(}\AttributeTok{width =}\NormalTok{ .}\DecValTok{4}\NormalTok{) }\SpecialCharTok{+}
      \FunctionTok{theme\_classic}\NormalTok{()}
\end{Highlighting}
\end{Shaded}

\subsection{Wide-form to Long-form Data:}\label{wide-form-to-long-form-data}

\begin{Shaded}
\begin{Highlighting}[]
\NormalTok{Wide\_data }\OtherTok{\textless{}{-}} \FunctionTok{read\_csv}\NormalTok{(“intervention.csv”)}

\FunctionTok{view}\NormalTok{(wide\_data)}
\end{Highlighting}
\end{Shaded}

\begin{Shaded}
\begin{Highlighting}[]
\FunctionTok{names}\NormalTok{(wide\_data)}

\NormalTok{long\_data }\OtherTok{\textless{}{-}}\NormalTok{ wide\_data }\SpecialCharTok{\%\textgreater{}\%}  
    \FunctionTok{pivot\_longer}\NormalTok{(}\AttributeTok{cols =} \FunctionTok{c}\NormalTok{(pre, post), }
    \AttributeTok{names\_to =} \StringTok{"time\_point"}\NormalTok{, }
    \AttributeTok{values\_to =} \StringTok{"sleep\_score"}\NormalTok{)}
\end{Highlighting}
\end{Shaded}

\chapter{Session 2: Correlation and Regression}\label{session-2-correlation-and-regression}

\chapter{Session 3: t-Tests and ANOVAs}\label{session-3-t-tests-and-anovas}

\end{document}
